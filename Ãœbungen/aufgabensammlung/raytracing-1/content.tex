\bivtask{Raytracing nach Whitted}{15}\label{aufgabe:raytracing}
%
Implementieren Sie rekursives Raytracing nach Algorithmus 10.5 und 10.6
des Skripts und verwenden Sie für das Auf\/finden der vom Strahl 
getroffenen Objekte den „naiven Ansatz“ von Seite 10-10.

Im Verzeichnis \bivfolder{/home/bildgen/Aufgaben/raytracing-1} finden Sie 
ein Archiv mit einem Rahmenprogramm, in dem Sie 
die Implementierung vornehmen können. Sie müssen nur Änderungen an der 
Datei \texttt{raytracer.cc} vornehmen. Compiliert wird das 
Rahmenprogramm durch Eingabe von \texttt{make}.

Folgende Methoden, Objekte und Strukturen werden für die Implementierung 
benötigt:
\begin{itemize}
  \item Die Klasse \texttt{Vector3d} stellt einen Vektor mit drei
        \texttt{double}-Komponenten dar und besitzt unter anderem die
        Methoden
        \begin{itemize}
          \item \texttt{cross(const Vector3d\&)} zur Berechnung des
                Kreuzproduktes mit einem weiteren Vektor,
          \item \texttt{dot(const Vector3d\&)} zur Berechnung des
                Innenproduktes mit einem weiteren Vektor,
          \item \texttt{norm()} liefert die euklidische Norm des Vektors 
                und
          \item \texttt{cwiseProduct(const Vector3d\&)} kann zur
                komponentenweisen Multiplikation mit einem weiteren 
                Vektor verwendet werden.
        \end{itemize}
        Außerdem sind die Operatoren \texttt{+}, \texttt{-}, \texttt{+=}
        sowie \texttt{-=} zur Addition und Subtraktion von Vektoren, die
        Operatoren \texttt{*} und \texttt{*=} zur Multiplikation mit
        Skalaren und die Operatoren \texttt{/} und \texttt{/=} zur 
        Division von Vektoren durch Skalare überladen.
  \item Die Klasse \texttt{Image} enthält das erzeugte Bild und stellt 
        die Methoden
        \begin{itemize}
          \item \texttt{getWidth()} und \texttt{getHeight()}
          \item sowie \texttt{setPixel(int x, int y, const Vector3d \&rgb)}
        \end{itemize}
        bereit. Verwenden Sie die Methode
        \texttt{Raytracer::clampCol(Vector3d\&)} vor dem Aufruf von
        \texttt{setPixel}, damit die Farbwerte auf das Intervall 
        $[0; 1]$ beschränkt werden.
  \item Die Klasse \texttt{Scene} enthält sämtliche die Szene
        betreffenden Informationen. Sie benötigen folgende Methoden und
        Members:
        \begin{itemize}
          \item \texttt{getCamera()} liefert ein Objekt der Klasse
                \texttt{Camera}, die weiter unten erläutert wird.
          \item \texttt{objects} ist ein Vektor, der Pointer auf alle in
                der Szene vorhanden Objekte vorhält.
          \item \texttt{getBackground()} liefert die Hintergrundfarbe 
                der Szene. 
          \item \texttt{lights} ist ein Vektor, der Pointer auf alle in 
                der Szene vorhanden Lichtquellen vorhält.
        \end{itemize}

  \item Sie benötigen folgende Methoden der Klasse \texttt{Camera}:
        \begin{itemize}
          \item \texttt{getPosition()} liefert die Position der Kamera 
                als \texttt{Vector3d}.
          \item \texttt{getDirection()} liefert die Blickrichtung der 
                Kamera als \texttt{Vector3d}.
          \item \texttt{getUp()} liefert den Up-Vektor der Kamera als 
                \texttt{Vector3d}.
          \item \texttt{getHorAngle()} liefert den horizontalen 
                Blickwinkel der Kamera als \texttt{double}.
      \end{itemize}
  \item Die verschiedenen Objekt-Typen sind von der Klasse
        \texttt{Object} abgeleitet und Sie benötigen:
        \begin{itemize}
          \item \texttt{isHitBy(const Ray\&)} gibt eine Instanz der 
                Struktur \texttt{HitInfo} zurück.
        \end{itemize}
  \item Alle Lichtquellen sind Instanzen der Klasse \texttt{Light} und 
        Sie benötigen:
        \begin{itemize}
          \item \texttt{visibleFrom(const Scene\&, const Vector3d\&)}
                bestimmt, ob die Lichtquelle vom übergebenen Punkt aus 
                sichtbar ist und gibt die Information in einer Instanz 
                der Struktur \texttt{VisibleInfo} zurück.
        \end{itemize}
  \item Die Struktur \texttt{Ray} stellt einen Strahl mit Ausgangspunkt 
        und Richtung dar. Alle Informationen dazu finden Sie in der 
        Datei \texttt{ray.h}.
  \item Die Struktur \texttt{HitInfo} bündelt die 
        Schnittpunkt-Informationen eines Strahls mit einem Objekt. Alle
        Informationen dazu finden Sie in der Datei \texttt{hitinfo.h}.
  \item Die Struktur \texttt{VisibleInfo} bündelt die Informationen,
        ob eine Lichtquelle von einem Punkt aus sichtbar ist. Alle
        Informationen dazu finden Sie in der Datei \texttt{visibleinfo.h}.
\end{itemize}
