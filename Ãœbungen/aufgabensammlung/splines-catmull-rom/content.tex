\bivtask{Catmull-Rom-Splines}{8}
%
Die Catmull-Rom-Splines können durch die Basismatrix
\[ M_{CR} = \frac12
\begin{pmatrix}
  -1 &  2 & -1 & 0 \\
   3 & -5 &  0 & 2 \\
  -3 &  4 &  1 & 0 \\
   1 & -1 &  0 & 0
\end{pmatrix} \]
definiert werden. Hierbei handelt es sich um einen interpolierenden
Spline, d.\,h.\ zu gegebenen Punkten $p₀, …, p_m$ werden
$p₁, …, p_{m-1}$ durch eine Kurve verbunden.
\begin{enumerate}
  \item Zeigen Sie, dass die Kurve wirklich durch $p₁, …, p_{m-1}$ 
        verläuft und damit die Stetigkeit.
  \item Berechnen Sie die Tangenten in den Punkten $pᵢ$,
        $i = 1, …, m - 1$, und zeigen Sie so, dass die Kurve $C¹$-stetig
        ist.
  \item Ist die Kurve $C²$-stetig? (Begründung!)
  \item Ergänzen Sie in Ihrer Lösung zu Aufgabe~\ref{aufgabe:bezier} die
        Funktion \texttt{maleCRSpline}.
\end{enumerate} 
