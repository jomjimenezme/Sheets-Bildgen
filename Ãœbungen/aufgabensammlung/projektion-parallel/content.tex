\bivtask{Parallelprojektion}{6}
%
Gegeben seien die folgenden Größen:
\[VRP = \pmat{3 \\ 1 \\ 10}, COP = \pmat{4 \\ 2 \\ 15},
VPN = \pmat{2 \\ 3 \\ 2}, VUP = \pmat{0 \\ 1 \\ 0},\]
\[u_\mathrm{min} = 0, u_\mathrm{max} = 40,
v_\mathrm{min} = 0, v_\mathrm{max} = 30, B = 0, F = 10.\]
Berechnen Sie die fünf für die Parallelprojektion benötigten Matrizen 
$T(-VRP)$, $R$, $SH_\mathrm{par}$, $T_\mathrm{par}$ und $S_\mathrm{par}$ 
und bestimmen Sie damit die Projektion des Punktes \\
\[Q = \pmat{10 \\ 10 \\ -10}\] in den kanonischen Bildraum.
