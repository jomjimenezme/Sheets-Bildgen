% Punktspiegelung im Raum
% Autor: Holger Arndt
% Datum: 10.11.2015
% Lösung zu Blatt 03, Aufgabe 09
\documentclass[a4paper,12pt]{scrartcl}
\usepackage{etex}
\usepackage{xltxtra}
\usepackage[babelshorthands]{polyglossia}
\usepackage{graphicx}
\usepackage{color}
\usepackage{hyperref}
\usepackage{amsmath}
\usepackage{unicode-math}
\setmathfont{XITS Math}

\typearea{18}
 
\pagestyle{empty}
\definecolor{linkcol}{rgb}{0.00,0.00,0.50}
\definecolor{urlcol}{rgb}{0.50,0.00,0.00}
\hypersetup{%
  pdftitle={Bildgenerierung, WS 2015/2016},
  pdfauthor={Holger Arndt},
  pdfpagemode=UseThumbs,
  colorlinks=true,
  linkcolor=linkcol,
  urlcolor=urlcol,
}

\setmainlanguage{german}
\defaultfontfeatures{Mapping=tex-text,Scale=MatchLowercase}
\setmainfont[Scale=1]{XITS}
\setsansfont[Scale=0.9]{FreeSans}
\setmonofont[Scale=0.82]{DejaVu Sans Mono}

\renewcommand{\theenumi}{\alph{enumi}}
\renewcommand{\labelenumi}{\theenumi)}

\newcommand{\pmat}[1]{\begin{pmatrix}#1\end{pmatrix}}

\begin{document}
 
\minisec{Punktspiegelung im Raum}

\begin{description}
\item[gegeben:] $P = \pmat{pₓ \\ p_y \\ p_z}$
\item[gesucht:] $Q = \pmat{qₓ \\ q_y \\ q_z} =
  \text{ Punktspiegelung von $P$ an $S = \pmat{a \\ b \\ c}$}$
\item[Punktspiegelung heißt:] $P - S = S - Q$
  \[ ⇒ Q = 2S - P = \pmat{2a - pₓ \\ 2b - p_y \\ 2c - p_z} \]
  \[ ⇒ T = \pmat{-1 & 0 & 0 & 2a \\ 0 & -1 & 0 & 2b \\ 0 & 0 & -1 & 2c \\ 0 & 0 & 0 & 1} \]
\end{description}

\end{document}
