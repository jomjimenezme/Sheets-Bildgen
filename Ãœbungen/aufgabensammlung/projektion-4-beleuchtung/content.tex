\bivtask{Färbung – Beleuchtungsmodell nach Phong}{4}\label{aufgabe:proj4}
%
Das Rahmenprogramm \texttt{proj4.cc} im Verzeichnis 
\bivfolder{/home/bildgen/Aufgaben/projektion-4} implementiert die 
perspektivische Projektion samt Füllung der Dreiecksflächen. Ergänzen 
Sie das Programm um eine pro Gitterpunkt berechnete Beleuchtung.
Implementieren Sie hierzu in der Funktion
\begin{alltt}
   VecRGB berechneBeleuchtung(const Vec3D& ecke, const Vec3D& normale,
                              const Vec3D& auge, const Vec3D& licht,
                              const DrawColour& farbe)
\end{alltt}
das Beleuchtungsmodell nach Phong für \emph{eine} Lichtquelle. Die 
Funktion wird für jeden der drei Eckpunkte einer Dreiecksfläche
aufgerufen und soll die Farbe an diesem Punkt berechnen. Die 
Interpolation der Farbe über die Fläche (Gouraud-Shading) übernimmt das 
Rahmenprogramm für Sie.

Die folgenden Größen sind vorgegeben:
\begin{center}
  \begin{tabular}{ll}
    \texttt{Vec3D ecke} & Position der Ecke in Weltkoordinaten \\
    \texttt{Vec3D normale} & Normale zur Fläche in dieser Ecke (in Weltkoordinaten) \\
    \texttt{Vec3D auge} & Koordinaten des Auges in Weltkoordinaten \\
    \texttt{Vec3D licht} & Koordinaten der Lichtquelle in Weltkoordinaten \\
    \texttt{VecRGB lightAmbient} & ambiente Lichtintensität \\
    \texttt{VecRGB lightDiffuse} & diffuse Lichtintensität \\
    \texttt{VecRGB lightSpecular} & Lichtintensität für winkelabhängige Reflexion \\
    \texttt{VecRGB materialAmbient} & ambienter Reflexionskoeffizient des Materials \\
    \texttt{VecRGB materialDiffuse} & diffuser Reflexionskoeffizient des Materials \\
    \texttt{VecRGB materialSpecular} & winkelabhängiger
            Reflexionskoeffizient des Materials \\
    \texttt{double materialSpecularity} & Exponent für winkelabhängige Reflexion \\
    \texttt{double c0, c1, c2} & Konstanten für entfernungsabhängige Dämpfung
  \end{tabular}
\end{center}

Die Rechnung mit Farbwerten innerhalb der Funktion erfolgt über
dreidimensionale Vektoren, deren Einträge die drei Kanäle des
RGB-Farbmodells darstellen. Die RGB-Intensitäten sind auf $[0, 1]$ 
normiert. Für Lichter stellen diese Werte die Lichtintensität des 
jeweiligen Farbkanals dar, für Materialien bzw. Objektoberflächen deren
Reflexionskoeffizienten für den jeweiligen Farbkanal.

Um nicht jeden Farbkanal einzeln berechnen zu müssen sind neben
\texttt{+} und \texttt{-} geeignete Operatoren vorgegeben,
z.\,B.\ \texttt{VecRGB v1 * VecRGB v2} für elementweise Multiplikation.

Beachten Sie, dass negative Skalarprodukte auf dem Auge bzw. der
Lichtquelle abgewandte Flächen hindeuten. Im Falle eines negativen 
Skalarproduktes ist der entsprechende Lichtanteil auf Null zu setzen, 
um keine negativen Lichtintensitäten zu erzeugen. Hierzu können sie die 
\texttt{max()}-Funktion benutzen.

Testen Sie Ihr Programm mit den \texttt{*.in}-Dateien.
