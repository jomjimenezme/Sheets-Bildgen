\bivtask{Perspektivische Projektion mit OpenGL}{2}
%
Im Verzeichnis \bivfolder{/home/bildgen/Aufgaben/opengl-2} finden Sie eine 
OpenGL-Implementierung der perspektivischen
Projektion aus Aufgabe~\ref{aufgabe:proj2}. Compilieren lässt sich das 
Programm mittels \texttt{make}. Weiterführende Informationen zu OpenGL 
finden Sie auf der Webseite\\
\href{http://www.opengl.org/sdk/docs/man/}{\texttt{http://www.opengl.org/sdk/docs/man/}}.

OpenGL nutzt im Wesentlichen zwei Matrizen, um Objekte, welche in 
Objektkoordinaten gegeben sind, in normalisierte Koordinaten zu 
transformieren und anschließend zu clippen und zu zeichnen. Zum ersten 
handelt es sich dabei um die sogenannte ModelView-Matrix, welche die 
Objektkoordinaten in Augenkoordinaten überführt. Diese Matrix übernimmt 
also die Transformation des Augenpunktes (gegeben durch \texttt{COP}) in
den Koordinatenursprung mit Blickrichtung (gegeben durch \texttt{-VPN}) 
entlang der negativen z-Achse, so wie es auf Seite 4-12 des Skriptes 
dargestellt ist. Die zweite Matrix ist die Projektionsmatrix, die die 
Augenkoordinaten im Unterschied zur Vorlesungen in einen Bildraum
\[B = \left\{(x,y,z)^T | -1\leq x\leq 1,-1\leq y\leq 1,-1\leq z\leq 1\right\}\] 
transformiert, an dem geclippt werden kann und welcher sich einfach auf 
den Bildschirm (den sogenannten Viewport) projizieren lässt.

Die Projektionsmatrix ist im Rahmenprogramm bereits mittels 
\texttt{glFrustum} gegeben und Ihre Aufgabe besteht zunächst darin, die 
ModelView-Matrix korrekt zu setzen. Verwenden Sie hierfür den Befehl 
\texttt{gluLookAt} der OpenGL Utility Library, deren Dokumentation Sie 
auch auf\\
\href{http://www.opengl.org/sdk/docs/man/}{\texttt{http://www.opengl.org/sdk/docs/man/}}

finden. Sobald dies implementiert ist, sollte Ihr Programm bereits---bis
auf das Clipping---die gleichen Bilder erzeugen, wie das Programm zu 
Aufgabe~\ref{aufgabe:proj2}. Das Clipping in z-Richtung funktioniert 
ebenfalls bereits.

Um nun identische Bilder zur Lösung von Aufgabe~\ref{aufgabe:proj2} zu 
erhalten, muss ein künstliches Clipping auf die durch 
\texttt{clip.uminf}, \texttt{clip.umaxf}, \texttt{clip.vminf} und 
\texttt{clip.vmaxf} gegebene Ausdehnung auf der Projektionsebene 
erfolgen. 

Ist die Projektionsebene gegeben durch 
$[\texttt{umin}, \texttt{umax}] \times [\texttt{vmin}, \texttt{vmax}]$, 
ergibt sich der geclippte Bereich als 
$[\texttt{umin} \cdot \texttt{clip.uminf}, \texttt{umax} \cdot \texttt{clip.umaxf}] \times [\texttt{vmin} \cdot \texttt{clip.vminf}, \texttt{vmax} \cdot \texttt{clip.vmaxf}]$.

Korrektes Clipping kann auf zwei Arten erzielt werden:
\begin{enumerate}
\item Mittels \texttt{glClipPlane} können Sie vier Clipping-Ebenen 
      definieren, die durch den Koordinatenursprung verlaufen und den 
      Sichtbereich in $x$- und $y$-Richtung auf der Projektionsebene auf 
      den oben erwähnten geclippted Bereich einschränken. Hierfür müssen 
      die Normalenvektoren der Ebenen in Abhängigkeit von \texttt{znear} 
      und der Größen des geclippten Bereiches ermittelt werden.
\item Das Sichtfenster (Viewport) kann entsprechend beschränkt werden. 
      Anschließend muss die Berechnung der Projektionsmatrix 
      (\texttt{glFrustum}) angepasst werden, so dass die Projektion nun 
      für den geclippten Bereich stattfindet.
\end{enumerate}

Testen Sie Ihr Programm mit den gleichen Eingabedateien wie auch 
Aufgabe~\ref{aufgabe:proj2}.
