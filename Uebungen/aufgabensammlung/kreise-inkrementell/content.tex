\bivtask{Scan Conversion für Kreise}{6}
%
Schreiben Sie eine Funktion
\begin{alltt}
   void drawCircle(Drawing& pic, IPoint2D center, int radius, bool filled,
                   int colour = 0)
\end{alltt}
die einen Kreis um den Punkt \texttt{center} mit Radius
\texttt{radius} zeichnet. Im Falle \texttt{filled} = \texttt{true}
soll der Kreis ausgefüllt werden. 

Implementieren Sie hierzu einen
\emph{inkrementellen, ganzzahligen Scan-Conversion-Algorithmus} zum
Zeichnen der Kreislinie und verwenden Sie eine Hilfsfunktion
\begin{alltt}
   void drawCirclePoints(Drawing& pic, int x, int y, IPoint2D center,
                         bool filled, int colour = 0)
\end{alltt}
welche zu einem Punkt $(\texttt{x}, \texttt{y})$ im zweiten Oktanten
alle hierdurch aus Symmetriegründen festgelegten Punkte
zeichnet oder verbindet, vergleiche Skizze.
\begin{center}
  \scalebox{0.9}{\input{kreise}}
\end{center}

Ein Rahmenprogramm finden Sie unter \bivfolder{/home/bildgen/Aufgaben/kreise}.
