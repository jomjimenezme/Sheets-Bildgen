\documentclass[a4paper,12pt]{scrartcl}
\usepackage{etex}
\usepackage{xltxtra}
\usepackage[babelshorthands]{polyglossia}
\usepackage{graphicx}
\usepackage{color}
\usepackage{hyperref}
\usepackage{amsmath}
\usepackage{unicode-math}
\setmathfont{XITS Math}

\usepackage{moreenum}

\typearea{18}
 
\pagestyle{empty}
\definecolor{linkcol}{rgb}{0.00,0.00,0.50}
\definecolor{urlcol}{rgb}{0.50,0.00,0.00}
\hypersetup{%
  pdftitle={Bildgenerierung, WS 2015/2016},
  pdfauthor={Holger Arndt},
  pdfpagemode=UseThumbs,
  colorlinks=true,
  linkcolor=linkcol,
  urlcolor=urlcol,
}

\setmainlanguage{german}
\defaultfontfeatures{Mapping=tex-text,Scale=MatchLowercase}
\setmainfont[Scale=1]{XITS}
\setsansfont[Scale=0.9]{FreeSans}
\setmonofont[Scale=0.82]{DejaVu Sans Mono}

\renewcommand{\theenumi}{\alph{enumi}}
\renewcommand{\labelenumi}{\theenumi)}

\newcommand{\pmat}[1]{\begin{pmatrix}#1\end{pmatrix}}

\begin{document}
 
\section*{Lösung zu Catmull-Rom-Splines –
  Stetigkeitsgrad}

% Betrachte jeweils nur die $x$-Koordinaten.

\begin{description}
\item[$C⁰$:] \mbox{}
  \[ Qᵢ(t) = \pmat{Pᵢ₋₃ & Pᵢ₋₂ & Pᵢ₋₁ & Pᵢ} · \frac12
  \pmat{-1 & 2 & -1 & 0 \\ 3 & -5 & 0 & 2 \\ -3 & 4 & 1 & 0 \\ 1 & -1 & 0 & 0} ·
  \pmat{t³ \\ t² \\ t \\ 1} \]
  \begin{align*}
%     xᵢ(1) &= \pmat{P_{i-3ₓ} & P_{i-2ₓ} & P_{i-1ₓ} & P_{iₓ}} ·
%     \pmat{0 \\ 0 \\ 1 \\ 0} = P_{i-1ₓ} \\
%     xᵢ₊₁(0) &= \pmat{P_{i-2ₓ} & P_{i-1ₓ} & P_{iₓ} & P_{i+1ₓ}} ·
%     \pmat{0 \\ 1 \\ 0 \\ 0} = P_{i-1ₓ}
    Qᵢ(1) &= \pmat{P_{i-3} & P_{i-2} & P_{i-1} & P_{i}} ·
    \pmat{0 \\ 0 \\ 1 \\ 0} = P_{i-1} \\
    Qᵢ₊₁(0) &= \pmat{P_{i-2} & P_{i-1} & P_{i} & P_{i+1}} ·
    \pmat{0 \\ 1 \\ 0 \\ 0} = P_{i-1}
  \end{align*}
\item[$C¹$:] \mbox{}
  \[ Qᵢ'(t) = \pmat{Pᵢ₋₃ & Pᵢ₋₂ & Pᵢ₋₁ & Pᵢ} · \frac12
  \pmat{-3 & 4 & -1 \\ 9 & -10 & 0 \\ -9 & 8 & 1 \\ 3 & -2 & 0} ·
  \pmat{t² \\ t \\ 1} \]
  \begin{align*}
%     xᵢ'(1) &= \pmat{P_{i-3ₓ} & P_{i-2ₓ} & P_{i-1ₓ} & P_{iₓ}} · \frac12
%     \pmat{0 \\ -1 \\ 0 \\ 1} = \frac12 \left(P_{iₓ} - P_{i-2ₓ}\right) \\
%     xᵢ₊₁'(0) &= \pmat{P_{i-2ₓ} & P_{i-1ₓ} & P_{iₓ} & P_{i+1ₓ}} · \frac12
%     \pmat{-1 \\ 0 \\ 1 \\ 0} = \frac12 \left(P_{iₓ} - P_{i-2ₓ}\right)
    Qᵢ'(1) &= \pmat{P_{i-3} & P_{i-2} & P_{i-1} & P_{i}} · \frac12
    \pmat{0 \\ -1 \\ 0 \\ 1} = \frac12 \left(P_{i} - P_{i-2}\right) \\
    Qᵢ₊₁'(0) &= \pmat{P_{i-2} & P_{i-1} & P_{i} & P_{i+1}} · \frac12
    \pmat{-1 \\ 0 \\ 1 \\ 0} = \frac12 \left(P_{i} - P_{i-2}\right)
  \end{align*}
\item[$C²$:] \mbox{}
  \[ Qᵢ''(t) = \pmat{Pᵢ₋₃ & Pᵢ₋₂ & Pᵢ₋₁ & Pᵢ} · \frac12
  \pmat{-6 & 4 \\ 18 & -10 \\ -18 & 8 \\ 6 & -2} ·
  \pmat{t \\ 1} \]
  \begin{align*}
%     xᵢ''(1) &= \pmat{P_{i-3ₓ} & P_{i-2ₓ} & P_{i-1ₓ} & P_{iₓ}} ·
%     \pmat{-1 \\ 4 \\ -5 \\ 2} = -P_{i-3ₓ} + 4P_{i-2ₓ} - 5P_{i-1ₓ} + 2P_{iₓ} \\
%     xᵢ₊₁''(0) &= \pmat{P_{i-2ₓ} & P_{i-1ₓ} & P_{iₓ} & P_{i+1ₓ}} ·
%     \pmat{2 \\ -5 \\ 4 \\ -1} = 2P_{i-2ₓ} - 5P_{i-1ₓ} + 4P_{iₓ} - P_{i+1ₓ}
    Qᵢ''(1) &= \pmat{P_{i-3} & P_{i-2} & P_{i-1} & P_{i}} ·
    \pmat{-1 \\ 4 \\ -5 \\ 2} = -P_{i-3} + 4P_{i-2} - 5P_{i-1} + 2P_{i} \\
    Qᵢ₊₁''(0) &= \pmat{P_{i-2} & P_{i-1} & P_{i} & P_{i+1}} ·
    \pmat{2 \\ -5 \\ 4 \\ -1} = 2P_{i-2} - 5P_{i-1} + 4P_{i} - P_{i+1}
  \end{align*}
  $C²$-stetig, falls $-Pᵢ₋₃ + 2Pᵢ₋₂ -  2Pᵢ + Pᵢ₊₁ = 0$, also im
  allgemeinen nicht.
\end{description}

\end{document}
