\bivtask{Rundflug um den Eiffelturm}{2}
%
Erzeugen Sie eine Animation, die einen spiralförmigen Rundflug um den 
Eiffelturm zeigt. Fahren Sie hierzu nicht nach Paris, sondern verwenden 
Sie die Daten aus der Datei \texttt{toureiffel.in}. Starten Sie Ihren 
Flug am Boden ($y = -662$) und fliegen Sie bis zur Spitze ($y=550$), 
wobei Sie den Turm einmal umrunden. Halten Sie einen konstanten Abstand 
von 800 Längeneinheiten von der Mitte des Turms ($y$-Achse) und blicken 
Sie immer waagerecht zur Mitte. Dann passen Ihre Aufnahmen auf Bilder 
der Größe $250 × 400$ Pixel. Der Abstand zur Projektionsebene spielt 
hier keine große Rolle und kann recht willkürlich gewählt werden. Setzen 
Sie Ihren Film aus 100 Einzelbildern zusammen.

Verwenden Sie die Eingabedatei \texttt{toureiffel.in} und ignorieren Sie 
die Werte der ersten beiden Zeilen (COP und TGT). Erzeugen Sie aus Ihrem 
Flug eine animierte \texttt{gif}- oder \texttt{mpg}-Datei. Ein 
Beispielprogramm zur Erzeugung von Animationen finden Sie im Verzeichnis
\bivfolder{/home/bildgen/Aufgaben/rundflug}. 
Zum Betrachten der Dateien können Sie \texttt{gwenview} verwenden, für 
\texttt{gif}-Ausgaben zusätzlich auch \texttt{eog}.

Falls Sie Aufgabe~\ref{aufgabe:proj4} nicht bearbeitet haben, lassen Sie 
die Beleuchtungsfunktion einfach die übergebene Farbe zurückgeben und 
setzen Sie den \texttt{outline} Parameter von \texttt{maleDreiecke} auf 
\texttt{true}.
