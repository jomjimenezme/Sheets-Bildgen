\bivtask{Bézier-Flächen mit OpenGL}{2}
%
Im Verzeichnis \bivfolder{/home/bildgen/Aufgaben/opengl-4} finden Sie eine 
OpenGL-Implementierung der Bézier-Flächen 
aus Aufgabe~\ref{aufgabe:bezierflaeche}. Ergänzen Sie in der Funktion
\begin{verbatim}
   void zeichneBezierFlaeche( const vector<vector<Vec3D> >& p,
                              int nMeshSize = 10 )
\end{verbatim}
das Zeichnen der Flächenstücke. Gehen Sie dafür wie folgt vor:
\begin{enumerate}
  \item Legen Sie mittels \texttt{glMap2f} und des Target-Parameters 
        \texttt{GL\_MAP2\_VERTEX\_3}, welcher dabei anzugeben ist, die 
        Kontrollpunkte des aktuellen Flächenstücks fest.
  \item Aktivieren Sie die Kontrollpunkte mittels \texttt{glEnable}.
  \item Erzeugen Sie unter Verwendung des Befehls \texttt{glMapGrid2f} 
        ein Mesh, das aus \texttt{nMeshSize} Partitionen in jeder 
        Richtung besteht.
  \item Zeichnen Sie die Bézier-Fläche mit \texttt{glEvalMesh2}.
\end{enumerate}

Im Gegensatz zu Aufgabe~\ref{aufgabe:bezierflaeche} wird hier nicht 
zwischen \texttt{anzkurv} und \texttt{anzlin} unterschieden sondern es
gibt nur einen Parameter \texttt{nMeshSize} für die Feinheit des Gitters.

Informationen zu den benötigten Befehlen erhalten Sie auf\\
\href{http://www.opengl.org/sdk/docs/man/}{\texttt{http://www.opengl.org/sdk/docs/man/}}
