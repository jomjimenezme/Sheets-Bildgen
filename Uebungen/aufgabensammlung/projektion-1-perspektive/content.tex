\bivtask{Perspektivische Projektion}{10}\label{aufgabe:proj1}
%
Programmieren Sie die perspektivische Projektion, die Überführung in
normalisierte Koordinaten sowie die Umwandlung in Gerätekoordinaten.
Ergänzen Sie hierzu im Rahmenprogramm \texttt{proj1.cc} im Verzeichnis
\bivfolder{/home/bildgen/Aufgaben/projektion-1} die entsprechenden Teile 
der Funktionen
\begin{alltt}
   Matrix4x4 berechneTransformation(const Vec3D& cop, const Vec3D& vrp,
                                    const Vec3D& vup, int w, int h,
                                    double& un, double& vn)
\end{alltt}
und
\begin{alltt}
   void maleLinien(Drawing& pic, const vector<Kante>& kanten,
                   const Matrix4x4& t, double un, double vn)
\end{alltt}
Alle für die Berechnung benötigten Operationen sind bereits vorgegeben, 
Sie können also z.\,B.\ die Matrix-Multiplikation direkt mittels 
\texttt{*}-Operator verwenden. Unterstützt werden ausschließlich 
4×4-Matrizen, geben Sie deshalb bei der Projektion und den nachfolgenden
Transformationen für die $n$-Koordinate eine Nullzeile in der Matrix an.

Gehen Sie für die Berechnung der Transformation in
\texttt{berechneTransformation()} in den folgenden Schritten vor und
erzeugen Sie zunächst jeweils einzelne Matrizen mit dem gewünschten
Effekt. \texttt{vpn}, \texttt{umin}, \texttt{umax}, \texttt{vmin} und
\texttt{vmax} sind bereits vorgegeben.

\begin{enumerate}
  \item Verschiebung des Aufpunktes der Projektionsebene (\texttt{vrp})
        in den Ursprung.
  \item Überführung in das $(u, v, n)$-Koordinatensystem.
  \begin{enumerate}
    \item Bestimmung des $(u, v, n)$-Koordinatensystems aus der
          Normalen der Projektionsebene, \texttt{vpn}, und der
          Aufwärtsrichtung, \texttt{vup}.
%     \item Rotation $n ↦ z$, $u ↦ x$, $v ↦ y$.
    \item Rotation $z ↦ n$, $x ↦ u$, $y ↦ v$.
  \end{enumerate}
  \item Verschieben der transformierten Augenposition (\texttt{cop}
        überführt in das $(u, v, n)$-Ko\-or\-di\-na\-ten\-sys\-tem) in
        den Koordinatenursprung. 
  \item Standard perspektivische Projektion auf die $(u, v)$-Ebene.
  \item Normalisierung der Koordinaten.
  \begin{enumerate}
    \item Translation der unteren linken Ecke des Projektionsfensters 
          in den Ursprung.
    \item Skalierung des Projektionsfensters, so dass es in das 
          Einheitsquadrat passt.
    \item Speichern der Ausdehnung des Bildes im Einheitsquadrat,
          $u_n$ und $v_n$ (der entsprechende Code-Abschnitt ist
          vorgegeben).
  \end{enumerate}
\end{enumerate}

Berechnen Sie nun die Gesamttransformation durch Matrix-Multiplikation 
in der richtigen Reihenfolge. Die Skalierung auf Gerätekoordinaten ist 
nicht Teil der Matrix, sie wird im Folgenden separat durchgeführt.
Praktischer Grund hierfür ist das einfache 3D-Clipping in normalisierten
Koordinaten.

Zum Zeichnen des Bildes ist dann noch Folgendes in \texttt{maleLinien()}
zu tun:
\begin{enumerate}
  \item Anwenden der Transformationsmatrix auf Anfangs- und Endpunkt der 
        einzelnen Linien (der entsprechende Code-Abschnitt ist 
        vorgegeben).
  \item Umwandlung der homogenen Koordinaten in 2D-Koordinaten.
  \item Skalierung auf Fenstergröße (Gerätekoordinaten) unter Verwendung 
        von $u_n$ und $v_n$.
\end{enumerate}

Testen Sie Ihre Transformation mit den \texttt{*.in}-Dateien,
z.\,B.\ „\texttt{proj1 < Colosseum.in}“.
