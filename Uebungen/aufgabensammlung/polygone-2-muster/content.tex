\bivtask{Gefüllte Polygone}{4}
%
Ändern Sie Ihre Polygon-Funktion aus Aufgabe~\ref{aufgabe:polygone}
(oder die Musterlösung, welche Sie im Verzeichnis
\bivfolder{/home/bildgen/Aufgaben/polygone-2} finden können) so ab, 
dass das Polygon nicht einfarbig, sondern mit einem Muster gefüllt wird. 
Ein Rahmenprogramm sowie Dateien mit Mustern und Beispieleingaben finden 
Sie im Verzeichnis \bivfolder{/home/bildgen/Aufgaben/polygone-2}.

Sorgen Sie dafür, dass die Füllung unabhängig davon ist, an welcher
Stelle des Bildes sich das Polygon befindet. D.h. das Polygon soll sich
wie ein bemaltes Stück Papier verhalten und nicht wie eine 
Loch-Schablone, die vor einem feststehenden bemalten Hintergrund bewegt 
wird. Dies können Sie mittels der Eingabedateien 
\texttt{polygmuster1.in} bis \texttt{polygmuster3.in} testen. In allen 
drei Füllungen sollte in der linken unteren Ecke des Dreiecks ein blaues
Quadrat des Musters zu sehen sein.

Der Befehl
\texttt{pic.drawPoint(x, y, QColor(muster.pixel(xmuster, ymuster)))}
malt einen Punkt in der Farbe, die das Muster \texttt{muster} an der
Stelle $(\mathtt{xmuster},\mathtt{ymuster})$ enthält, an die Stelle 
$(\mathtt{x},\mathtt{y})$ ins Bild \texttt{pic}. Mit 
\texttt{muster.width()} und \texttt{muster.height()} können Sie Breite 
und Höhe des Musters ermitteln.
