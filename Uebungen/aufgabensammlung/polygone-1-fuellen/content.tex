\bivtask{Gefüllte Polygone}{10}\label{aufgabe:polygone}
%
Schreiben Sie eine Funktion
\begin{alltt}
   void drawFilledPolygon(Drawing& pic, const vector<IPoint2D>& ecken,
                          int colour = 0)
\end{alltt}
die mittels Algorithmus 3.22 der Vorlesung ein ausgefülltes Polygon 
zeichnet. Durchlaufen Sie die Zeilen $y_{min}$ bis $y_{max}$, die das 
Polygon enthalten, und datieren Sie in jedem Schritt eine 
\emph{Tabelle der aktiven Kanten} auf. Dies ist eine Liste, die zu jeder 
Kante, die die aktuelle $y$-Scan-Line schneidet, deren oberen Endpunkt, 
den aktuellen $x$-Wert sowie die Steigung enthält. Die Kanten werden 
nach $x$ und ggf.\ nach Steigung sortiert gespeichert.

Im Verzeichnis \bivfolder{/home/bildgen/Aufgaben/polygone-1} finden Sie ein 
Rahmenprogramm und einige Eingabedateien.
