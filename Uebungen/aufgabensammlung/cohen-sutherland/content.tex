\bivtask{Strecken-Clipping nach Cohen und Sutherland}{4}\label{aufgabe:cohen-sutherland}
%
Sei das Kappungsfenster gegeben durch das achsenparallele Rechteck
$[2; 8] × [1; 5]$ sowie Linien mit Anfangspunkt $P₁$ und Endpunkt $P₂$.

\begin{minipage}{0.3\textwidth}
  \begin{enumerate}
    \item[a)] $P₁ = \pmat{1 \\ 2}$, $P₂ = \pmat{10 \\ 4}$
    \item[c)] $P₁ = \pmat{7 \\ 0}$, $P₂ = \pmat{10 \\ 2}$ % Tippfehler: sollte 10 \\ 2 sein
  \end{enumerate}
\end{minipage}%
\begin{minipage}{0.4\textwidth}
  \begin{enumerate}
    \item[b)] $P₁ = \pmat{3 \\ 3}$, $P₂ = \pmat{6 \\ 0}$
    \item[d)] $P₁ = \pmat{5 \\ 6}$, $P₂ = \pmat{11 \\ 4}$
  \end{enumerate}
\end{minipage}

\vspace{1ex}
Bestimmen Sie mit dem Cohen-Sutherland-Verfahren jeweils Anfangs- und 
Endpunkt der gekappten Linie von Hand.
