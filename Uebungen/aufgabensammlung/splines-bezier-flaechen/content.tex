\bivtask{B\'{e}zier-Flächen}{10}\label{aufgabe:bezierflaeche}
%
Ergänzen Sie eine Funktion \texttt{berechneBezierFlaeche}, welche die 
Kanten der B\'{e}zier-Fläche bestimmt, im Rahmenprogramm 
\texttt{surfaces.cc} unter \bivfolder{/home/bildgen/Aufgaben/flaechen}.

\begin{center}
  \includegraphics*[scale=0.5]{surfaceBsp.png}
\end{center}

Zu jeweils 16 Punkten
$\text{\texttt{p[i][j]}}, \ldots, \text{\texttt{p[i + 3][j + 3]}}$ 
gehört ein Flächenstück bestehend aus \texttt{anzkurv} Kurvenstücken für 
jede der beiden Richtungen, wobei jedes Kurvenstück durch 
\texttt{anzlin} Linien approximiert wird.

Mit dem Befehl \texttt{vk.push\_back(Kante(anf, end, BLACK));} können 
Sie dem Kanten-Vektor die einzelnen Kantenstücke hinzufügen, wobei 
\texttt{anf} und \texttt{end} vom Typ \texttt{Vec4D} sind.
