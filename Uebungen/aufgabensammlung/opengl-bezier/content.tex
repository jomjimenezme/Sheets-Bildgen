\bivtask{Bézier-Kurven mit OpenGL}{4}
%
Im Verzeichnis \bivfolder{/home/bildgen/Aufgaben/opengl-6} finden Sie eine 
OpenGL-Implementierung der Bézier-Kurven 
aus Aufgabe~\ref{aufgabe:bezier}. Ergänzen Sie in der Funktion
\begin{verbatim}
   void maleBezier( const vector<DPoint2D>& points, int nSegments )
\end{verbatim}
das Zeichnen der Kurvenstücke. Gehen Sie dafür wie folgt vor:
\begin{enumerate}
  \item Legen Sie mittels \texttt{glMap1f} und des Target-Parameters 
        \texttt{GL\_MAP1\_VERTEX\_3}, welcher dabei anzugeben ist, die 
        Kontrollpunkte des aktuellen Kurvenstücks fest.
  \item Aktivieren Sie die Kontrollpunkte mittels \texttt{glEnable}.
  \item Teilen Sie OpenGL mit, dass durch Linien verbundene Punkte
        gezeichnet werden sollen. Dies erfolgt mit dem Befehl 
        \texttt{glBegin( GL\_LINE\_STRIP );}
  \item Werten Sie die Bézier-Kurve mittels \texttt{glEvalCoord1f} an 
        Zwischenpunkten aus, so dass pro Kurvenstück \texttt{nSegments} 
        Linien entstehen.
  \item Beenden Sie das Zeichnen mit \texttt{glEnd}.
\end{enumerate}

Informationen zu den benötigten Befehlen erhalten Sie auf\\
\href{http://www.opengl.org/sdk/docs/man/}{\texttt{http://www.opengl.org/sdk/docs/man/}}

Sie können zum Testen die Dateien \texttt{points?.in} benutzen.
