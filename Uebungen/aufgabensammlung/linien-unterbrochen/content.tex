\bivtask{Unterbrochene Linien}{6}
%  
Erweitern Sie das Rahmenprogramm \texttt{ulinien.cc}, welches Sie im 
Verzeichnis \bivfolder{/home/bildgen/Aufgaben/ulinien} finden können, so 
dass unterbrochene Linien gezeichnet werden. Dabei ist 
\texttt{std::vector<float> mask} die Maske, die definiert, an welchen 
Stellen die Linie unterbrochen wird.

Beachten Sie, dass die gegebene Maske nur für horizontale und vertikale 
Linien verwendet werden kann. Überlegen Sie sich, wie Sie eine 
steigungsunabhängige Strichlänge realisieren können und implementieren 
Sie Ihr Programm entsprechend. Das Beispielbild im Verzeichnis
\bivfolder{/home/bildgen/Aufgaben/ulinien} zeigt, wie 
eine Implementierung mit steigungsunabhängigen Strichlängen aussehen 
\emph{kann}.
