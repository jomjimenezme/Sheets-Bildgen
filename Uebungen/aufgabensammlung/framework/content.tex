\bivtask{Test des Frameworks}{0}
%
Für die Bearbeitung der Übungsblätter existiert auf den Rechnern des
CIP-Clusters in Raum G.14.11 ein Qt-Framework. Dieses befindet sich im 
Verzeichnis
%
\vspace{-.5em}%
\begin{alltt}
   /home/bildgen/cppqt
\end{alltt}
\vspace{-.5em}%
Um damit arbeiten zu können, gehen Sie wie folgt vor:
%
\begin{bivsubt}
	\bivitem Falls in Ihrem Home-Verzeichnis nicht bereits ein
		Unterverzeichnis mit dem Namen \texttt{bin} existiert, erstellen
		Sie es mittels
		\begin{alltt}
		   cd ~
		   mkdir bin 
		\end{alltt}
		Sie müssen sich danach neu einloggen, damit das Verzeichnis dem
		Suchpfad hinzugefügt wird.
	\bivitem Erzeugen Sie einen symbolischen Link auf das Compiler-Script
		\texttt{g++drawqt} in Ihrem soeben erzeugten Verzeichnis 
		\texttt{\string~/bin}:
		\begin{alltt}
		   ln -s /home/bildgen/cppqt/g++drawqt \string~/bin
		\end{alltt}
	\bivitem Ein Beispielprogramm finden Sie unter
		\begin{alltt}
		   /home/bildgen/cppqt/bsp1.cc
		\end{alltt}
		Kopieren Sie das Beispielprogramm in Ihr Home-Verzeichnis und
		rufen Sie das Compiler-Script mit der Quelltext-Datei des
		Beispiels als Parameter auf, um es zu kompilieren:
		\begin{alltt}
		   cp /home/bildgen/cppqt/bsp1.cc .
		   g++drawqt bsp1.cc 
		\end{alltt}
		Erlaubte Dateiendungen sind \texttt{c}, \texttt{cc}, \texttt{cpp} 
		und \texttt{c++}.
	\bivitem Das erzeugte Programm trägt den Namen der Quelltextdatei 
		ohne deren Endung. Testen Sie das Beispielprogramm:
		\begin{alltt}
		   ./bsp1
		\end{alltt}
\end{bivsubt}
%
Eine vollständige Auf\/listung der Funktionalität des Qt-Frameworks 
finden Sie unter:
%
\vspace{-.5em}%
\begin{alltt}
   /home/bildgen/cppqt/doc/index.html 
\end{alltt}
\vspace{-.5em}%
%
Es existiert noch ein weiteres Compiler-Script namens 
\texttt{g++drawqt-g}, welches mit Debug-Symbolen compiliert.
