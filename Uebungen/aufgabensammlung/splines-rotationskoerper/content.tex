\bivtask{Rotationskörper}{10}\label{aufgabe:rotk}
%
Ergänzen Sie die Funktion \texttt{berechneRotationsKoerper}, welche die
Kanten der (vertikalen) B\'{e}zier-Splines und die Kanten der (horizontalen)
Kreise des Rotationskörpers bestimmt, im Rahmenprogramm \texttt{rotk.cc}
unter \bivfolder{/home/bildgen/Aufgaben/rotationskoerper}.

\begin{center}
  \includegraphics*[scale=0.5]{rotkBsp.png}    
\end{center}

Zu jeweils 4 Punkten
$\text{\texttt{p[i]}}, \ldots, \text{\texttt{p[i + 3]}}$ in der
$x$-$y$-Ebene gehören
\begin{itemize}
  \item \texttt{anzkurv} Bézierkurven, die durch \texttt{anzlinku}
        Geradenstücke approximiert werden und durch Drehung der
        "`Originalkurve"' um die $x$-Achse entstehen,
  \item \texttt{anzkreis} Kreise um die $x$-Achse, die durch
        \texttt{anzlinkr} Geradenstücke angenähert werden.
\end{itemize}
Beachten Sie, dass die $x$-Achse hier ausnahmsweise nach oben zeigt. Wie 
in Aufgabe~\ref{aufgabe:bezierflaeche} sollen alle Geradenstücke in 
einen Kantenvektor eingefügt werden.
