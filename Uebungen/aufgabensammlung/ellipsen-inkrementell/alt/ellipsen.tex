\documentclass[12pt]{scrartcl}

\usepackage{amsmath}
\usepackage{hyperref}
\usepackage[utf8]{inputenc}

\pagestyle{empty}

\setlength{\parindent}{0em}

\begin{document}

\begin{center}
  {\textbf{\Large Inkrementelle Scan Conversion für Ellipsen}\\ -- Alte Version --}\\ 
\end{center}
\medskip

\section*{Region 1}

Wir starten in Region 1 am Punkt $(x,y)=(0,b)$ und stellen fest, dass
\begin{align*} 
  F(x, y) & = b^2 x^2 + a^2 y^2 - a^2 b^2\\
          & = a^2 b^2 - a^2 b^2\\
          & = 0.
\end{align*}
Wir müssen untersuchen, ob
\begin{align*} 
  F(x+1, y-\frac{1}{2}) & = b^2 (x+1)^2 + a^2 (y-\frac{1}{2})^2 - a^2 b^2\\
                        & = b^2 x^2 + 2 b^2 x + b^2 + a^2 y^2 - a^2 y + \frac{1}{4} a^2 - a^2 b^2\\
                        & =: d_1
\end{align*}
$\leq 0$, $>0$ oder $=0$ ist. Da wir an Punkt $(x,y)=(0,b)$ starten, können wir $d_1$ also mit
\begin{align*}
  d_1 = b^2 - a^2 b + \frac{1}{4} a^2
\end{align*}
initialisieren.
Die Frage ist nun, wie sich $d_1$ ändert, wenn wir die Entscheidung $d_1 \leq 0$ (Punkt in Ellipse), also nach {\em Osten} treffen, bzw. die Entscheidung zu $d_1>0$ (Punkt außerhalb Ellipse), also nach {\em Südosten}, ausfällt.
Dafür untersuchen wir im Fall {\em Ost} die Differenz
\begin{align*}
  d_{1,O}(x,y) & = F(x+2,y-\frac{1}{2}) - F(x+1,y-\frac{1}{2})                           \\
               & =   (b^2 x^2 + 4 b^2 x + 4 b^2 + a^2 y^2 - a^2 y + \frac{1}{4} a^2 - a^2 b^2)\\
               & \phantom{=} - (b^2 x^2 + 2 b^2 x + b^2 + a^2 y^2 - a^2 y + \frac{1}{4} a^2 - a^2 b^2)\\
               & =   2 b^2 x + 3 b^2.
\end{align*}
Im Fall {\em Südost} müssen wir die Differenz
\begin{align*}
  d_{1,SO}(x,y) & = F(x+2,y-\frac{3}{2}) - F(x+1,y-\frac{1}{2})                           \\
                & =   (b^2 x^2 + 4 b^2 x + 4 b^2 + a^2 y^2 - 3 a^2 y + \frac{9}{4} a^2 - a^2 b^2)\\
                & \phantom{=} - (b^2 x^2 + 2 b^2 x + b^2 + a^2 y^2 - a^2 y + \frac{1}{4} a^2 - a^2 b^2)\\
                & =   2 b^2 x + 3 b^2 - 2 a^2 y + 2 a^2
\end{align*}
berechnen.
Um einen ganzzahligen Algorithmus zu erhalten müssen wir einfach $d_1, d_{1,O}(x,y)$ und $d_{1,SO}(x,y)$ mit $4$ multiplizieren.
Wir müssen also im Fall {\em Ost} $d_1 \leftarrow d_1 + d_{1,O}(x,y)$ und im Fall {\em Südost} $d_1 \leftarrow d_1 + d_{1,SO}(x,y)$ setzen.


\section*{Region 2}
Während des Zeichnens von Region~1 wird geprüft, ob die Bedingung
\[
  a^2 y > b^2 x
\]
erfüllt ist.
Sobald sie nicht mehr erfüllt ist, findet der Übergang in Region~2 statt.
Nun muss neu initialisiert werden.
Dafür setzen wir 
\begin{align*}
  F(x+\frac{1}{2}, y-1) & = b^2 (x+\frac{1}{2})^2 + a^2 (y-1)^2 - a^2 b^2\\
                        & =: d_2
\end{align*}
und für den Punkt $(x,y)$ wählen wir den zuletzt gesetzten Punkt.
Danach muss eine Entscheidung für {\em Südost} oder {\em Süd} getroffen werden.
Im Fall {\em Südost} bilden wir die Differenz
\begin{align*}
  d_{2,SO}(x,y) & = F(x+\frac{3}{2},y-2) - F(x+\frac{1}{2},y-1)                           \\
                & = (b^2 x^2 + 3 b^2 x + \frac{9}{4} b^2 + a^2 y^2 - 4 a^2 y + 4 a^2 - a^2 b^2)\\
                & \phantom{=} - (b^2 x^2 + b^2 x + \frac{1}{4} b^2 + a^2 y^2 - 2 a^2 y + a^2 - a^2 b^2)\\
                & =   2 b^2 x + 2 b^2 - 2 a^2 y + 3 a^2.
\end{align*}
Für die Entscheidung {\em Süd} untersuchen wir die Differenz
\begin{align*}
  d_{2,S}(x,y) & = F(x+\frac{1}{2},y-2) - F(x+\frac{1}{2},y-1)                           \\
               & = (b^2 x^2 + b^2 x + \frac{1}{4} b^2 + a^2 y^2 - 4 a^2 y + 4 a^2 - a^2 b^2)\\
               & \phantom{=} - (b^2 x^2 + b^2 x + \frac{1}{4} b^2 + a^2 y^2 - 2 a^2 y + a^2 - a^2 b^2)\\
               & =   - 2 a^2 y + 3 a^2.
\end{align*}
Genau wie in Region~1 können $d_2, d_{2,SO}$ und $d_{2,S}$ mit $4$ multipiziert werden, um einen ganzzahligen Algorithmus zu erhalten.


\section*{Differenzen 2. Ordnung}

Die Differenzen $d_{1,O}, d_{1,SO}, d_{2,SO}$ und $d_{2,S}$ sind noch von $x$ bzw.\ $y$ abhängig.
Dies kann aber vermieden werden, wenn die Änderung dieser Größen untersucht wird.
Man muss ausrechnen, wie sich z.B.\ $d_{1,O}$ ändert, wenn die Entscheidung nach {\em Ost} ($(x,y)\rightarrow(x+1,y)$) oder {\em Südost} ($(x,y)\rightarrow(x+1,y-1)$) getroffen wird.
In Region~1 erhält man dann
\begin{align*}
  d_{1,O,O}   & := d_{1,O}(x+1,y)    - d_{1,O}(x,y)  = 2 b^2\\
  d_{1,SO,O}  & := d_{1,SO}(x+1,y)   - d_{1,SO}(x,y) = 2 b^2\\
  d_{1,O,SO}  & := d_{1,O}(x+1,y-1)  - d_{1,O}(x,y)  = 2 b^2\\
  d_{1,SO,SO} & := d_{1,SO}(x+1,y-1) - d_{1,SO}(x,y) = 2 b^2 + 2 a^2.
\end{align*}
Analog dazu erhält man in Region~2
\begin{align*}
  d_{2,SO,SO} & := d_{2,SO}(x+1,y-1) - d_{2,SO}(x,y) = 2 b^2 + 2 a^2\\
  d_{2,S,SO}  & := d_{2,S}(x+1,y-1)  - d_{2,S}(x,y)  = 2 a^2\\
  d_{2,SO,S}  & := d_{2,SO}(x,y-1)   - d_{2,SO}(x,y) = 2 a^2\\
  d_{2,S,S}   & := d_{2,S}(x,y-1)    - d_{2,S}(x,y)  = 2 a^2.
\end{align*}
Um einen ganzzahligen Algorithmus zu erhalten, müssen auch diese Werte alle mit $4$ multipliziert werden.
Man könnte nun also $d_1, d_{1,O}$ und $d_{1,SO}$ initialisieren und je nach Richtungsentscheidung setzt man z.B.
\begin{align*}
  d_1      & \leftarrow d_1 + d_{1,O}\\
  d_{1,O}  & \leftarrow d_{1,O} + d_{1,O,O}\\
  d_{1,SO} & \leftarrow d_{1,SO} + d_{1,SO,O}
\end{align*}
nach der Entscheindung {\em Ost} oder
\begin{align*}
  d_1      & \leftarrow d_1 + d_{1,SO}\\
  d_{1,O}  & \leftarrow d_{1,O} + d_{1,O,SO}\\
  d_{1,SO} & \leftarrow d_{1,SO} + d_{1,SO,SO}
\end{align*}
nach der Entscheindung {\em Südost} in Region~1.




\end{document}
