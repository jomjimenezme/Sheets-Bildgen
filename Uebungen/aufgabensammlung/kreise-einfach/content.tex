\bivtask{Einfache Kreise}{3}
%
Schreiben Sie ein Programm, das
%
\begin{enumerate}
  \item ein zunächst leeres Bild erzeugt,
  \item einen Mittelpunkt $m$ und einen Radius $r$ einliest und
  \item einen \emph{gefüllten} Kreis um $m$ mit Radius $r$ malt.
\end{enumerate}
%
Verwenden Sie nicht den optimierten (inkrementellen) Kreisalgorithmus,
sondern lassen Sie ihren Algorithmus auf den \emph{naiven} Varianten aus
dem Skript (oder ggf. einer eigenen naiven Variante) beruhen.
Kommentieren Sie in Ihrem Programm, wie der Algorithmus funktioniert.

Verwenden Sie nur die Funktion
\texttt{Drawing::drawPoint()} und \emph{nicht}
\texttt{Drawing::drawCircle()}.
