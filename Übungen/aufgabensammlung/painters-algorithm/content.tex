\bivtask{Painter's Algorithm}{10}\label{aufgabe:painters}
%
Der \emph{Painter's Algorithm} ist ein besonders einfacher Algorithmus 
zur Elimination verdeckter Kanten und Flächen. Er ist anwendbar, wenn 
die Objekte sich nicht gegenseitig durchdringen, und basiert auf 
folgender Strategie: Man zeichnet einfach \emph{alle} Objekte (Polygone 
müssen dabei ausgefüllt werden), wobei \emph{„von hinten nach vorne“} 
gearbeitet wird. Dadurch werden die durch weiter vorne liegende Objekte 
verdeckten (und damit unsichtbaren) Teile der hinteren Objekte 
automatisch „übermalt“.

Stellen Sie mit Hilfe des Painter's Algorithm den Graphen der Funktion
\[ f(s, t) = 10 e^{-\left(\frac{s²}{25}+\frac{t²}{100}\right)} \] auf 
dem Bereich $(s, t) ∈ [-5; 5] × [-5; 5] $ dar.

Approximieren Sie dabei den Graphen durch ein geeignetes Netz von 
Dreiecken und verwenden Sie zur Darstellung die sogenannte 
\emph{Kabinett-Projektion}.

\begin{center}
  \unitlength 0.7cm
  \begin{picture}(4,4)(-1,-1)
  \thicklines
  \put(0,0){\vector(1,0){2}}
  \put(2,0.2){\shortstack{$s$}}
  \put(0,0){\vector(0,1){2}}
  \put(0.2,2){\shortstack{$f$}}
  \put(0,0){\vector(-2,-1){0.866}}
  \put(-0.4,-1){\shortstack{$t$}}
  \end{picture}
\end{center}

Dabei handelt es sich um eine Parallelprojektion, wobei die $s$- und die
$f$-Achse waagerecht bzw.\ senkrecht dargestellt werden und die 
$t$-Achse um 30° geneigt und um den Faktor ½ verkürzt ist.

Verwenden Sie das Rahmenprogramm \texttt{painters.cc} unter 
\bivfolder{/home/bildgen/Aufgaben/painters} und implementieren Sie die 
Funktionen:
\begin{alltt}
   Matrix4x4 berechneMpar(double umin, double umax, double vmin, double vmax,
                          double& ratio)
   void erzeugeFlaeche(double xmin, double xmax, double zmin, double zmax, int num,
                       const std::function<double(double, double)>& func,
                       std::vector<Dreieck>& dreiecke )
\end{alltt}
