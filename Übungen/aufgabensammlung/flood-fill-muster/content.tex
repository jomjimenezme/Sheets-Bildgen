\bivtask{Parallelogramm und Kreis mit Muster füllen}{8}
%
Schreiben Sie Routinen
\begin{verbatim}
   void parallelogramm( Drawing &pic, IPoint2D P1, IPoint2D P2,
                        IPoint2D P3, char muster,
                        const DrawColour &farbe1,
                        const DrawColour &farbe2 )
\end{verbatim}
und
\begin{verbatim}
   void kreis( Drawing &pic, IPoint2D M, int radius, char muster,
               const DrawColour &farbe1, const DrawColour &farbe2 )
\end{verbatim}
zum Zeichnen ausgefüllter Parallelogramme und Kreise.

Dabei seien $\mathtt{P_{1}} = (x1, y1)$, $\mathtt{P_{2}} = (x2, y2)$, 
$\mathtt{P_{3}} = (x3, y3)$ drei aufeinanderfolgende Ecken des 
Parallelogramms, wenn dieses {\em entgegen dem Uhrzeigersinn\/} 
durchlaufen wird, \texttt{M} sei der Mittelpunkt des Kreises und 
\texttt{radius} sein Radius.

Sehen Sie die Möglichkeit vor, die Objekte mit vorgegebenen Mustern zu 
füllen. Als Muster sollten wenigstens die folgenden symbolischen Werte 
zur Verfügung stehen:
\begin{itemize}
  \item \texttt{[V]OLL}: vollständig gefüllt mit {\texttt farbe1}
  \item \texttt{[N]O}: "`Nord-Ost"'-schraffiert (Linien der Steigung 
        $+1$ mit {\texttt farbe1})
  \item \texttt{[S]O}: "`Süd-Ost"'-schraffiert (Linien der Steigung 
        $-1$ mit {\texttt farbe1})
  \item \texttt{[K]AROS}: "`Nord-Ost"'- und "`Süd-Ost"'-schraffiert.
\end{itemize}
Achten Sie darauf, dass das Muster {\em translationsinvariant\/} ist, 
d.h. beim Verschieben des Objekts wird auch das Muster "`mitbewegt"'.

Für die nicht zum Muster gehörigen Pixel soll {\texttt farbe2} verwendet 
werden.

Das Muster soll nicht vorberechnet werden. Stattdessen soll anhand der 
Position beim Zeichnen entschieden werden, ob ein Pixel im Objekt zur 
Schraffur ({\texttt farbe1}) oder zum Hintergrund ({\texttt farbe2}) 
gehört.

\textbf{Hinweise:}
\begin{itemize}
  \item Ein Rahmenprogramm und Beispiel-Eingabedateien für diese Aufgabe 
        finden Sie im Verzeichnis \bivfolder{/home/bildgen/Aufgaben/flood-fill}.
  \item Zur Bestimmung der {\em Ränder\/} der Objekte können Sie Ihre 
        Lösungen zu den Aufgaben~5 und 6 des dritten Blattes verwenden.
  \item Wenn Sie Aufgabe 5 und 6 nicht bearbeitet haben, können Sie hier 
        auch die Funktionen \texttt{Drawing::drawCircle()} und 
        \texttt{Drawing::drawLine()} verwenden.
  \item Zum Füllen verwenden Sie einen modifizierten (rekursionsfreien) 
        Flood Fill-Ansatz.
  \item Verwenden Sie die Funktion 
        \texttt{Drawing::getPointColour(int x, int y)}, die eine Instanz
        von \texttt{DrawColour} zurückgibt, um zu testen, welche Farbe 
        ein Pixel des Bildes hat.
\end{itemize}
