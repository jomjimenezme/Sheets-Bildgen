\bivtask{Gefüllte Polygone}{12}
%
Schreiben Sie eine Funktion
\begin{alltt}
   void drawPatternPolygon( Drawing& pic, const vector<IPoint2D>& ecken,
                            const QImage& muster )
\end{alltt}
die mittels Algorithmus 3.22 der Vorlesung ein mit einem Muster 
gefülltes Polygon zeichnet.

\textbf{Hinweise:}
\begin{itemize}
  \item Im Verzeichnis \bivfolder{/home/bildgen/Aufgaben/polygone-3} auf 
        dem CIP-Cluster finden Sie ein Rahmenprogramm und 
        Beispiel-Eingabedateien.
  \item Schreiben Sie zunächst eine Funktion, welche das Polygon mit 
        einer beliebigen Farbe füllt und modifizieren Sie diese 
        anschließend, so dass mit einem Muster gefüllt wird.
  \item Durchlaufen Sie die Zeilen $y_{min}$ bis $y_{max}$, die das 
        Polygon enthalten, und datieren Sie in jedem Schritt eine 
        \emph{Tabelle der aktiven Kanten} auf. Es ist hilfreich, dies 
        als eine Liste zu implementieren, die zu jeder Kante, welche die 
        aktuelle $y$-Scan-Line schneidet, deren oberen Endpunkt, den
        aktuellen $x$-Wert sowie die Steigung enthält. Die Kanten werden 
        nach $x$ und ggf.\ nach Steigung sortiert gespeichert.
  \item Sorgen Sie dafür, dass die Füllung translationsinvariant, also
        unabhängig davon ist, an welcher Stelle des Bildes sich das 
        Polygon befindet.
  \item Der Befehl \texttt{pic.drawPoint(x, y, QColor(muster.pixel(xmuster,ymuster)))}
        malt im Bild \texttt{pic} an die Stelle $(\mathtt{x},\mathtt{y})$
        einen Punkt in der Farbe, die das Muster \texttt{muster} an der
        Stelle $(\mathtt{xmuster},\mathtt{ymuster})$ enthält. Breite und
        Höhe des Musters können Sie mit \texttt{muster.width()} und 
        \texttt{muster.height()} ermitteln.
\end{itemize} 
