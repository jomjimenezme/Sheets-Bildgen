\bivtask{Einfache Linien}{3}
%
Schreiben Sie ein Programm, das
%
\begin{enumerate}
  \item ein zunächst leeres Bild erzeugt,
  \item\label{it:punkteLesen} zwei Punkte einliest,
  \item\label{it:linieZeichnen} eine Linie zeichnet, die beide Punkte
        verbindet,
  \item und \ref{it:punkteLesen} bis \ref{it:linieZeichnen} solange
        wiederholt, bis der Anwender negative Koordinaten eingibt.
\end{enumerate}
%
Implementieren Sie zum Zeichnen der Linien den \emph{naiven}
Algorithmus aus dem Skript (oder ggf. einen eigenen naiven Algorithmus)
aber noch nicht die optimierten (inkrementellen) Varianten.
Kommentieren Sie in Ihrem Programm, wie der Algorithmus funktioniert.

Verwenden Sie hierzu nur die Funktion
\texttt{Drawing::drawPoint()} und \emph{nicht}
\texttt{Drawing::drawLine()}.
