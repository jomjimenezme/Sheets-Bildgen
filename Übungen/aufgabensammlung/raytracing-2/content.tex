\bivtask{Raytracing nach Whitted - Teil 2}{8}
%
Erweitern Sie die Implementierung zum rekursiven Raytracing aus 
Aufgabe~\ref{aufgabe:raytracing} um Lichtbrechung und die Darstellung
von Tetraedern.

Im Verzeichnis \bivfolder{/home/bildgen/Aufgaben/raytracing-2} finden Sie 
ein ein Rahmenprogramm, in dem Sie die 
Implementierung vornehmen können. Sie müssen nur Änderungen an den 
markierten Stellen in den Dateien \texttt{raytracer.cc}, 
\texttt{triangle.cc} und \texttt{tetraeder.cc} vornehmen.
Orientieren Sie sich für die Implementierung der Klassen 
\texttt{Tetraeder} und \texttt{Triangle} an den Klassen \texttt{Box} 
und \texttt{Parallelogram}. In der Datei \texttt{README} wird 
beschrieben, wie Sie das Rahmenprogramm compilieren können. 
% Eine Einweisung erfolgt außerdem in der Übung.

Zusätzlich zu den erläuterten Methoden, Objekten und Strukturen aus 
Aufgabe~\ref{aufgabe:raytracing} benötigen Sie die Methoden
\begin{itemize}
  \item \texttt{Object::getRefraction()}, die den Brechungsindex für das 
        Medium des Objekts als \texttt{double} liefert, und
  \item \texttt{Object::getPigment()}, die einen \texttt{Vector3d} 
        zurückgibt und die Durchlässigkeit des Objekts für die drei 
        Farbkomponenten darstellt.
\end{itemize}
  
