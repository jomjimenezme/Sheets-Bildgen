\newcommand{\XXX}{\typeout{Hier fehlt
    etwas!}$▶◀$}


\newenvironment{srcana}{%
  \minisec{Analysis of the source code}\nopagebreak
  \begin{itemize}}{%
  \end{itemize}}

%\newfont{\ttnarr}{pcrr8tn scaled 1728}
%\newfontfamily\ttnarr[Scale=0.9]{DejaVu Sans Mono}
\newcommand{\hreftt}[2]{\href{#1}{\texttt{#2}}}
%\newcommand{\hreftt}[2]{\href{#1}{\scalebox{0.9}[1]{\texttt{#2}}}}

\newcommand{\N}{ℕ}
\newcommand{\Q}{ℚ}
\newcommand{\R}{ℝ}
\newcommand{\Z}{ℤ}

\newcommand{\disjcup}{\stackrel{\makebox[0mm][c]{\raisebox{-1.2ex}[0mm][0mm]{%
        $\cdot$}}}{\cup}}
\newcommand{\bigdisjcup}[1][]{\stackrel{\makebox[0mm][c]{%
      \raisebox{-2.2ex}[0mm][0mm]{$\cdot$}}}{\bigcup_{#1}}}

\newcounter{savecounter}

\newcommand{\BIGskip}{\Bigskip\Bigskip}
\newcommand{\Bigskip}{\bigskip\bigskip}
\newcommand{\negskip}{\vspace*{-1em}}
\newcommand{\Negskip}{\negskip\negskip}
\newcommand{\NEGskip}{\Negskip\negskip}

%\newcommand{\entspr}{\ensuremath{\;\widehat{=}\;}\xspace}
\newcommand{\imp}{\ensuremath{\Rightarrow}}
\renewcommand{\iff}{\ensuremath{\Leftrightarrow}}
%\newcommand{\contradict}{\ensuremath{\lightning}\xspace}
%\newcommand{\contradict}{☇}
\newcommand{\contradict}{⚡}
\newcommand{\Good}{\textcolor{mygreen}{⊕}}
\newcommand{\Bad}{\textcolor{myred}{⊖}}

\newcommand{\dH}{d.\,h.\xspace}
\newcommand{\Dh}{D.\,h.\xspace}
\newcommand{\iA}{i.\,Allg.\xspace}
\newcommand{\IA}{I.\,Allg.\xspace}
\newcommand{\oae}{o.\,ä.\xspace}
\newcommand{\so}{s.\,o.\xspace}
\newcommand{\sog}{sog.\xspace}
\newcommand{\su}{s.\,u.\xspace}
\newcommand{\ua}{u.\,a.\xspace}
\newcommand{\uA}{u.\,A.\xspace}
\newcommand{\uU}{u.\,U.\xspace}
\newcommand{\uae}{u.\,ä.\xspace}
\newcommand{\vChr}{v.\,Chr.\xspace}
\newcommand{\nChr}{n.\,Chr.\xspace}
\newcommand{\zB}{z.\,B.\xspace}
\newcommand{\ZB}{Z.\,B.\xspace}
\newcommand{\zT}{z.\,T.\xspace}
\newcommand{\bzw}{bzw.\xspace}
\newcommand{\usw}{usw.\xspace}
\newcommand{\evtl}{evtl.\xspace}
\newcommand{\Evtl}{Evtl.\xspace}
\newcommand{\bzgl}{bzgl.\xspace}
\newcommand{\ggf}{ggf.\xspace}
\newcommand{\Ggf}{Ggf.\xspace}
\newcommand{\vgl}{vgl.\xspace}
\newcommand{\Vgl}{Vgl.\xspace}
\newcommand{\oBdA}{o.\,B.\,d.\,A.\xspace}
\newcommand{\OBdA}{O.\,B.\,d.\,A.\xspace}
\newcommand{\inkl}{inkl.\xspace}
\newcommand{\cf}{cf.\xspace}
\newcommand{\Cf}{Cf.\xspace}
\newcommand{\eg}{e.\,g.\xspace}
\newcommand{\Eg}{E.\,g.\xspace}
\newcommand{\ie}{i.\,e.\xspace}
\newcommand{\Ie}{I.\,e.\xspace}
\newcommand{\etc}{etc.\xspace}
\newcommand{\wLog}{w.\,l.\,o.\,g.\xspace}
\newcommand{\Wlog}{W.\,l.\,o.\,g.\xspace}
\newcommand{\wrt}{w.\,r.\,t.\xspace}
\newcommand{\resp}{resp.\xspace}
\newcommand{\glqq}{„}
\newcommand{\grqq}{“}
\newcommand{\glq}{‚}
\newcommand{\grq}{‘}

% \let\origTextcolor\textcolor
% %\newcommand{\myTextcolor}{\setboolean{notblackif}{true}\origTextcolor}
% \DeclareRobustCommand\myTextcolor{\@ifnextchar[\myTextcolorA\myTextcolorB}
% % \newcommand{\myTextcolorA}[3][]{%
% %   \setboolean{notblackif}{true}\origTextcolor[#1]{#2}{#3}\setboolean{notblackif}{false}}
% %  \newcommand{\myTextcolorB}[2]{%
% %   \setboolean{notblackif}{true}\origTextcolor{#1}{#2}\setboolean{notblackif}{false}}
% \newcommand{\myTextcolorA}[3][]{%
%   \begingroup\renewcommand{\mathcol}{}\origTextcolor[#1]{#2}{#3}\endgroup}
% \newcommand{\myTextcolorB}[2]{%
%   \begingroup\renewcommand{\mathcol}{}\origTextcolor{#1}{#2}\endgroup}
% \let\textcolor\myTextcolor
\newcommand{\textcolorrgbif}[2]{\ifthenelse{%
    \boolean{colourif}}{%
    \textcolor[rgb]{#1}{#2}}{%
    \textcolor[rgb]{0,0,0}{#2}}}
\newcommand{\black}[1]{\textcolor{black}{#1}}
\newcommand{\white}[1]{\textcolor{white}{#1}}
\newcommand{\mygreen}[1]{\textcolor{mygreen}{#1}}
\newcommand{\myred}[1]{\textcolor{myred}{#1}}
\newcommand{\myblue}[1]{\textcolor{myblue}{#1}}
\newcommand{\mymagenta}[1]{\textcolor{mymagenta}{#1}}
\newcommand{\mycyan}[1]{\textcolor{mycyan}{#1}}
\newcommand{\myyellow}[1]{\textcolor{myyellow}{#1}}
\newcommand{\mylightblue}[1]{\textcolor{mylightblue}{#1}}
\newcommand{\mdef}[1]{\textcolor{mdefcol}{#1}}
\newcommand{\cmt}[1]{\textcolor{cmtcol}{(#1)}} % \Comment bei Bruno
%\newcommand{\cmt}[1]{(#1)} % falls kein Farbwechsel gewünscht
\newcommand{\xfigBlue}[1]{\textcolor{xfigBlue}{#1}}
\newcommand{\xfigBlueIII}[1]{\textcolor{xfigBlueIII}{#1}}
\newcommand{\xfigBrownII}[1]{\textcolor{xfigBrownII}{#1}}
\newcommand{\xfigBrownIII}[1]{\textcolor{xfigBrownIII}{#1}}
\newcommand{\xfigCyanIII}[1]{\textcolor{xfigCyanIII}{#1}}
\newcommand{\xfigGreen}[1]{\textcolor{xfigGreen}{#1}}
\newcommand{\xfigGreenIII}[1]{\textcolor{xfigGreenIII}{#1}}
\newcommand{\xfigGreenIV}[1]{\textcolor{xfigGreenIV}{#1}}
\newcommand{\xfigMagentaII}[1]{\textcolor{xfigMagentaII}{#1}}
\newcommand{\xfigMagentaIII}[1]{\textcolor{xfigMagentaIII}{#1}}
\newcommand{\xfigMagentaIV}[1]{\textcolor{xfigMagentaIV}{#1}}
\newcommand{\xfigRed}[1]{\textcolor{xfigRed}{#1}}
\newcommand{\xfigRedII}[1]{\textcolor{xfigRedII}{#1}}
\newcommand{\xf}[1]{%
  \ifthenelse{%
    \boolean{colourif}}{%
    #1}{%
    \textcolor{black}{#1}}
}
\newcommand{\interactgrave}{\textcolor{lstinteractcol}{̀ }}

\newcommand{\shortEx}[1]{%
  \,\textcolor{excol}{\mytextrm{\textit{(#1)}}}\,}

\newcommand{\mytextrm}[1]{{\myroman #1}}
\newcommand{\irm}[1]{\mytextrm{#1}} % Kurzfassung, für inkscape
\newcommand{\irmit}[1]{\mytextrm{\textit{#1}}} % Kurzfassung, für inkscape
\newcommand{\itt}[1]{\texttt{#1}} % Kurzfassung, für inkscape
\newcommand{\iisc}[2]{%
  \DIVIDE{1}{#1}{\tmpsc}%
  \scalebox{\tmpsc}{#2}
} % inverse Skalierung für inkscape

%\newfont{\symbolfont}{pzdr}
\newcommand{\yes}{\mygreen{\myromanalt ✓}}

\newcommand{\mathtxtit}[1]{\textit{\mdseries\myromanmath #1}}
\newcommand{\mathtxtrm}[1]{\textrm{\mdseries\myromanmath #1}}

\newcommand{\emoji}[1]{{\emojifont #1}}

\newcommand{\rk}[1]{\left(#1\right)}
\newcommand{\fk}[1]{\mleft(#1\mright)}
\newcommand{\tstk}[1]{\!\left(#1\right)}%to be removed after testing
\newcommand{\ek}[1]{\left[#1\right]}
\newcommand{\erk}[1]{\left[#1\right)}
\newcommand{\rek}[1]{\left(#1\right]}
\newcommand{\mk}[1]{\left\{#1\right\}}
\newcommand{\clk}[1]{\left\lceil #1\right\rceil}
\newcommand{\flk}[1]{\left\lfloor #1\right\rfloor}
\newcommand{\angk}[1]{\left\langle #1\right\rangle}
\newcommand{\abs}[1]{\left\lvert#1\right\rvert}
\newcommand{\norm}[2][]{\left\lVert#2\right\rVert_{#1}}
\newcommand{\restrict}[2]{\left.#1\right|_{#2}}
\newcommand{\pmat}[1]{\begin{pmatrix}#1\end{pmatrix}}
\newcommand{\bigrk}[1]{\bigl(#1\bigr)}
\newcommand{\Bigrk}[1]{\Bigl(#1\Bigr)}
\newcommand{\set}[2]{\left\{#1\mathrel{}\middle|\mathrel{}#2\right\}}
\def\xminusfill@{\arrowfill@{}\relbar{}}
\newcommand{\xminus}[2][]{%
  \ext@arrow 0099\xminusfill@{#1}{\raisebox{0.8ex}[0mm][0mm]{$\scriptstyle#2$}}}

\newcommand{\person}[9]{%
  \hfill \raisebox{#1}[1mm][0mm]{\scriptsize
    \begin{tabular}[b]{r}
      #2 \\
      \ifthenelse{\equal{#3}{}}{}{* #3  \\}
      \ifthenelse{\equal{#4}{}}{}{† #4  \\}
      #5 \\[-1mm]
      \ifthenelse{\equal{#7}{}}{}{\scalebox{0.5}{Foto: #7} \\[-2mm]}
      \ifthenelse{\equal{#8}{}}{}{\scalebox{0.5}{Quelle: #8} \\[-2mm]}
      \ifthenelse{\equal{#9}{}}{}{\scalebox{0.5}{Lizenz: #9} \\[-2mm]}
    \end{tabular}
    \includegraphics[height=20mm]{#6}}
}
\newcommand{\personHidden}[9]{%
  \hfill \raisebox{#1}[1mm][0mm]{\scriptsize
    \begin{tabular}[b]{r}
      #2 \\
      \ifthenelse{\equal{#3}{}}{}{* #3  \\}
      \ifthenelse{\equal{#4}{}}{}{† #4  \\}
      #5 \\[-1mm]
      \ifthenelse{\equal{#7}{}}{}{\begin{WebHide}\scalebox{0.5}{Foto:
            #7}\end{WebHide}
      \\[-2mm]}
      \ifthenelse{\equal{#8}{}}{}{\begin{WebHide}\scalebox{0.5}{Quelle:
            #8}\end{WebHide}
      \\[-2mm]}
      \ifthenelse{\equal{#9}{}}{}{\begin{WebHide}\scalebox{0.5}{Lizenz:
            #9}\end{WebHide}
      \\[-2mm]}
    \end{tabular}
    \ifthenelse{\boolean{webif}}{\rule{0.2mm}{20mm}}{\includegraphics[height=20mm]{#6}}}
}
\newcommand{\imglic}[4]{\tiny%
  \ifthenelse{\equal{#1}{}}{}{{Foto: #1},}
  \ifthenelse{\equal{#2}{}}{}{{Quelle: #2},}
  \ifthenelse{\equal{#3}{}}{}{{Titel: #3},}
  \ifthenelse{\equal{#4}{}}{}{{Lizenz: #4}}
}
\newcommand{\grflic}[4]{\tiny%
  \ifthenelse{\equal{#1}{}}{}{{Grafik: #1},}
  \ifthenelse{\equal{#2}{}}{}{{Quelle: #2},}
  \ifthenelse{\equal{#3}{}}{}{{Titel: #3},}
  \ifthenelse{\equal{#4}{}}{}{{Lizenz: #4}}
}

%\DeclareUrlCommand\urlsf{\urlstyle{sf}}%no link!
\newcommand{\urlsf}[1]{{\urlstyle{sf}\url{#1}}}
\newcommand{\urlpers}[1]{{\urlstyle{sf}\url{#1}}}
\newcommand{\urllic}[1]{{\urlstyle{sf}\url{#1}}}
% %allow breaks in URL, cf. https://tex.stackexchange.com/questions/3033/forcing-linebreaks-in-url
% \expandafter\def\expandafter\UrlBreaks\expandafter{\UrlBreaks%  save the current one
%   \do\a\do\b\do\c\do\d\do\e\do\f\do\g\do\h\do\i\do\j%
%   \do\k\do\l\do\m\do\n\do\o\do\p\do\q\do\r\do\s\do\t%
%   \do\u\do\v\do\w\do\x\do\y\do\z\do\A\do\B\do\C\do\D%
%   \do\E\do\F\do\G\do\H\do\I\do\J\do\K\do\L\do\M\do\N%
%   \do\O\do\P\do\Q\do\R\do\S\do\T\do\U\do\V\do\W\do\X%
%   \do\Y\do\Z}

\newcommand{\exlsg}[1]{%
  \begin{description}
  \item[\color{excol}Lösung:] #1
  \end{description}}
\newcommand{\exsol}[1]{%
  \begin{description}
  \item[\color{excol}Solution:] #1
  \end{description}}
\newcommand{\algimpl}[1]{%
  \marginpar{{\makebox[0mm][r]{\footnotesize $\to$ \AnyLink{#1}}}}}

\DeclareMathOperator{\ggT}{ggT}
\DeclareMathOperator{\id}{id}
\DeclareMathOperator{\round}{round}

\newcolumntype{C}{>{$}c<{$}}
\newcolumntype{L}{>{$}l<{$}}
\newcolumntype{R}{>{$}r<{$}}
